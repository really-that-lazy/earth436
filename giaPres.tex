\documentclass{article}

\usepackage{graphicx}

\usepackage[%
  papersize={12.8cm,9.6cm},
  hmargin=1cm,%
  vmargin=1cm,%
  head=0.5cm,% might be changed later
  headsep=0pt,%
  foot=0.5cm% might be changed later
  ]{geometry}% http://ctan.org/pkg/geometry

\begin{document}
\newpage
   %\vspace*{\stretch{1.0}}
   \begin{center}
      \Large{\textbf{\\Obtaining rates of glacial isostatic adjustment from Unequally spaced data}}
      \newline
      \large{\\John Lawson}
      \Large\textit{\\University of Waterloo Earth Sciences Honours Thesis}
   \end{center}
   \vspace*{\stretch{2.0}}

\newpage



\section{Introduction}
-Movement of Earths crust atop the mantle is driven by many factors including buoyancy\\
-Addition and removal of weight, in this case the Laurentide Ice Sheet, from the crust will cause vertical adjustments known as GIA\\
-As inclination of ground surface changes, water levels (ie Lakes) and the paths taken by
the flow of water (rivers, lake outlets, groundwater) change\\
-This has implications for engineering and environmental assessments\\
-projection of future changes due to GIA relies on having a reliable estimate of
past rates of GIA\\
% cant infer long term process from short term data
% mention past efforts using water gauges, gps data
\newpage

\section[2]{Previous Work}
-OSL dating used to determine age for sequences of Quaternary beach deposits (strandplain sequences) at each site vs current day elevation.\\
-This is then used to create a graph of elevation vs time for each site\\
-Paleohydrographs were created in Johnston et al. 2012
\newpage
\section{Locations}   
-Grand Traverse Bay, Michigan (GTB)\\
-Au Train Bay, Michigan (ATB)\\
-Batchawana Bay, Ontario (BATB)\\
-Tahquamenon Bay, Michigan (TAHB)\\
\\
\begin{figure}[h]
	\makebox[\textwidth]{\includegraphics[width=0.72\paperwidth]{johnstonLaurentianMap.png}}
	\caption{Map of the study sites used. Reproduced with permission from Johnston et al. 2012}
	\label{fig:jj2012map}
\end{figure}

\newpage


% add map, maybe with data points
\newpage
% talk about location
% talk about link between importance of water levels and they cant be understood
% without understanding gia
% talk about what gia is
\begin{figure}[h]
	\includegraphics[width=1.1\linewidth]{data/theDataRaw.png}
	\caption{Current day elevation of relict shorelines with respect to time before present over the last 5000 years. Strandplain sites Au Train Bay, Michigan (ATB), Batchawana Bay, Ontario (BATB), Tahquamenon Bay, Michigan (TAHB), and Grand Traverse Bay, Michigan (GTB) surrounding Lake Superior are plotted individually. Data from Johnston et al, (2012)}
	\label{fig:rawData}
\end{figure} 
%\begin{figure}[h]
	\includegraphics[width=0.40\textwidth]{data/legendary.png}
	%\makebox[\textwidth]{\includegraphics[width=\paperwidth]{data/legendary.png}}
	%\caption{Site Legend}
	%\label{fig:rdmLegend}
\end{figure}


% talk about structure of data, what are axes, why is it increasing
% talk about objective
\newpage
\section[2]{Method}
-Johnston et al. 2012 used a linear regression method for each site to get relative GIA rates between sites\\
-Model is likely too simplistic and doesnt take account of local variations\\
-Data not available continuously for each site, or at the same time for each site to compare gia, so values must be inferred using linear interpolation between known data points.\\
\begin{figure}[h]
	\includegraphics[width=1.1\linewidth]{data/theData.png}
	\caption{Water surface elevation with respect to time before present, modelled}
	\label{fig:rawDataWithModel}
\end{figure}
\newpage
%\begin{figure}[h]
	\includegraphics[width=0.40\textwidth]{data/legendary.png}
	%\makebox[\textwidth]{\includegraphics[width=\paperwidth]{data/legendary.png}}
	%\caption{Site Legend}
	%\label{fig:rdmLegend}
\end{figure}


% note mainville & craymer, previous work has used water level gauge data, but
% cant infer a long term process from short term data
\newpage
\section[2]{Method}
-GIA is now plotted by subtracting between the measured values of one dataset and the modelled value of another.\\
-6 combinations of pairs of sites are created, each of which has a forward (A to B) and backward (B to A) comparison.\\
\newpage
%\section{Previous Work} 
Mainville \& Craymer (2005) used water gauge data collected around the LGL over the past 150 years to
 create monthly means of water level. Differences in these values between sites
 were then plotted against time to calculate a rate of elevation change between
 sites over time (This value is interpreted to represent the impact of the GIA
 process on the crust underlying the LGL, even though the actual process extends
 over a much longer timescale than that of the data collection). Combinations of sites were shown to produce
 inconsistent results, so a second method using a least squares adjustment process was used,
 removing some monthly mean outliers which plotted at or beyond some arbitrary residual distance away 
 from the linear regression line in the vertical (elevation) axis. 
 This process was repeated with each new linear regression on the remaining data
 points until none remained "too far away" from
 the final regression line. A third, and ultimately optimal method for calculating
 GIA was developed by
 Mainville \& Craymer in their 2005 paper, using the original
 method of directly comparing monthly
 water level means, but this time with adjustments for the epoch, site, and month of the year.
 Their findings with this method showed a general agreement with the post glacial
 ICE-3G global model of GIA at that time, while the ICE-4G model developed by Peltier
 was shown to underestimate the relative difference in vertical movement across
 the span of the Great Lakes (Mainville \& Craymer, 2005).\\ \\
Johnston et al. (2012) attempted to provide a value for GIA in the LGL with
 better accuracy than previous estimates calculated using water
 gauge data.
 
In order to accomplish this, the data used to measure the process of
 GIA needed to extend over a much longer timescale. In this method, water
 levels were inferred from the elevation of relict shorelines in beach ridge
 strandplains from the late Holocene sediment record surrounding Lake Superior.
 Ages for each elevation were inferred from age dating samples from these beach
 deposits (known as strandplain sequences) using
 optically stimulated luminescence (OSL) age dating. Johnston et al (2012) differed from
 Mainville \& Craymer (2005), in that data collected for the 2012 paper using OSL
 age dating did not have
 elevations sampled at the same points in time for calculation of relative
 rates. As a result, Johnston et al (2012) the elevation vs time data was modelled with a linear
 regression for each site, the difference in slopes of each regression representing the GIA rate
 between sites. Individual regressions were further created per site
 for a series of four ranges of time related to lake level phases, namely the Nipissing,
 Algoma, Sault, and Sub-Sault (Johnston et al, 2012). The results reported from this process
 are summarized in Figure \ref{fig:jj2012Grid}. \\
 
 \begin{figure}[t]
	\includegraphics[width=0.9\linewidth]{jjGrid.png}
	\caption{GIA values reported by Johnston et al 2012. All values are in cm/century.}
	\label{fig:jj2012Grid}
 \end{figure}
 % need to ask JJ about JJ 2012, bottom of page 3, divergence of intercepts
 



\newpage
\begin{figure}[t]
	\makebox[\textwidth]{\includegraphics[width=0.72\paperwidth]{data/ATB-BATB_DataAndModel.png}}
	\caption{ATB-BATB raw data with linear interpolation model}
	\label{fig:data_ATBxBATB}
\end{figure}
\newpage

\begin{figure}[t]
	\includegraphics[width=0.9\linewidth]{data/gias/theGIA_ATB_relative_to_BATB.png}
	\caption{Differences in elevation measured from the ATB data to the ATB model}
	\label{fig:gias_ATBxBATB}
\end{figure}
\newpage


\begin{figure}[t]
	\includegraphics[width=0.9\linewidth]{data/gias/theGIA_BATB_relative_to_ATB.png}
	\caption{Differences in elevation measured from the BATB data to the ATB model}
	\label{fig:gias_BATBxATB}
\end{figure}
\newpage
% this desperately needs to be done as a loop



\begin{figure}[h]
	\makebox[\textwidth]{\includegraphics[width=0.72\paperwidth]{data/TAHB-BATB_DataAndModel.png}}
	\caption{TAHB-BATB raw data with linear interpolation model}
	\label{fig:data_TAHBxBATB}
\end{figure}
\newpage

\begin{figure}[h]
	\includegraphics[width=0.9\linewidth]{data/gias/theGIA_TAHB_relative_to_BATB.png}
	\caption{Differences in elevation measured from the TAHB data to the BATB model}
	\label{fig:gias_TAHBxBATB}
\end{figure}
\newpage


\begin{figure}[h]
	\includegraphics[width=0.9\linewidth]{data/gias/theGIA_BATB_relative_to_TAHB.png}
	\caption{Differences in elevation measured from the BATB data to the TAHB model}
	\label{fig:gias_BATBxTAHB}
\end{figure}
\newpage








\begin{figure}[h]
	\makebox[\textwidth]{\includegraphics[width=0.72\paperwidth]{data/TAHB-ATB_DataAndModel.png}}
	\caption{TAHB-ATB raw data with linear interpolation model}
	\label{fig:data_TAHBxATB}
\end{figure}
\newpage

\begin{figure}[h]
	\includegraphics[width=0.9\linewidth]{data/gias/theGIA_TAHB_relative_to_ATB.png}
	\caption{Differences in elevation measured from the TAHB data to the ATB model}
	\label{fig:gias_TAHBxATB}
\end{figure}
\newpage


\begin{figure}[h]
	\includegraphics[width=0.9\linewidth]{data/gias/theGIA_ATB_relative_to_TAHB.png}
	\caption{Differences in elevation measured from the ATB data to the TAHB model}
	\label{fig:gias_ATBxTAHB}
\end{figure}
\newpage






\begin{figure}[h]
	\makebox[\textwidth]{\includegraphics[width=0.72\paperwidth]{data/GTB-BATB_DataAndModel.png}}
	\caption{GTB-BATB raw data with linear interpolation model}
	\label{fig:data_GTBxBATB}
\end{figure}
\newpage

\begin{figure}[h]
	\includegraphics[width=0.9\linewidth]{data/gias/theGIA_GTB_relative_to_BATB.png}
	\caption{Differences in elevation measured from the GTB data to the BATB model}
	\label{fig:gias_GTBxBATB}
\end{figure}
\newpage


\begin{figure}[h]
	\includegraphics[width=0.9\linewidth]{data/gias/theGIA_BATB_relative_to_GTB.png}
	\caption{Differences in elevation measured from the BATB data to the GTB model}
	\label{fig:gias_BATBxGTB}
\end{figure}
\newpage









\begin{figure}[h]
	\makebox[\textwidth]{\includegraphics[width=0.72\paperwidth]{data/GTB-ATB_DataAndModel.png}}
	\caption{GTB-ATB raw data with linear interpolation model}
	\label{fig:data_GTBxATB}
\end{figure}
\newpage

\begin{figure}[h]
	\includegraphics[width=0.9\linewidth]{data/gias/theGIA_GTB_relative_to_ATB.png}
	\caption{Differences in elevation measured from the GTB data to the ATB model}
	\label{fig:gias_GTBxATB}
\end{figure}
\newpage


\begin{figure}[h]
	\includegraphics[width=0.9\linewidth]{data/gias/theGIA_ATB_relative_to_GTB.png}
	\caption{Differences in elevation measured from the ATB data to the GTB model}
	\label{fig:gias_ATBxGTB}
\end{figure}
\newpage










\begin{figure}[h]
	\makebox[\textwidth]{\includegraphics[width=0.72\paperwidth]{data/GTB-TAHB_DataAndModel.png}}
	\caption{GTB-TAHB raw data with linear interpolation model}
	\label{fig:data_GTBxTAHB}
\end{figure}
\newpage

\begin{figure}[h]
	\includegraphics[width=0.9\linewidth]{data/gias/theGIA_GTB_relative_to_TAHB.png}
	\caption{Differences in elevation measured from the GTB data to the TAHB model}
	\label{fig:gias_GTBxTAHB}
\end{figure}
\newpage


\begin{figure}[h]
	\includegraphics[width=0.9\linewidth]{data/gias/theGIA_TAHB_relative_to_GTB.png}
	\caption{Differences in elevation measured from the TAHB data to the GTB model}
	\label{fig:gias_TAHBxGTB}
\end{figure}
\newpage


\begin{figure}[h]
	\makebox[\textwidth]{\includegraphics[width=0.6\paperwidth]{data/intervals.png}}
	\caption{95p Confidence intervals on GIA rates obtained from site comparisons}
	\label{fig:intervalsGIA}
\end{figure}


\newpage


\newpage
\begin{figure}[h]
	\makebox[\textwidth]{\includegraphics[width=0.72\paperwidth]{johnstonLaurentianMapWithMyGIARates.png}}
	\caption{Relative GIA Rates produced by this papers method}
	\label{fig:myGIARates}
\end{figure}
\newpage
\begin{figure}[h]
	\makebox[\textwidth]{\includegraphics[width=0.72\paperwidth]{mainvilleGias.png}}
	\caption{Relative GIA Rates produced by Mainville \& Craymer (reproduced from Mainville \& Craymer, 2005)}
	\label{fig:craymerGIARatesBigPlot}
\end{figure}
\newpage
\begin{figure}[h]
	\makebox[\textwidth]{\includegraphics[width=0.72\paperwidth]{johnstonLaurentianMapWithCraymerGIARates.png}}
	\caption{Relative GIA Rates produced by Mainville \& Craymer}
	\label{fig:craymerGIARates}
\end{figure}



\end{document}
