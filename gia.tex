\documentclass{article}

\usepackage{graphicx}

\begin{document}
\newpage
   %\vspace*{\stretch{1.0}}
   \begin{center}
      \Large{\textbf{\\Obtaining rates of glacial isostatic adjustment from Unequally spaced data}}
      \newline
      \large{\\John Lawson}
      \Large\textit{\\University of Waterloo Earth Sciences Honours Thesis}
   \end{center}
   \vspace*{\stretch{2.0}}

\newpage


\section{Abstract}
The continental crust underlying the Great Lakes is currently rebounding upwards
 after being depressed by the weight of ice sheets formed in the most recent glacial period.
 The rate of rebound varies by location, and exerts a significant control on the flow of water
 in the area as the inclination of the ground surface changes with it. In order to predict the
 future movement in this area, the rate of glacial isostatic adjustment (GIA) must be
 inferred from measurements of the water level in the geological record. These measurements are
 made by age dating series of beach deposits using optically stimulated luminescence (OSL).\\
 
 The key focus of this paper is to properly analyze this data, by projecting water levels
 between the measured data points, then comparing relative water levels of the 4 different sites
 to create a plot of relative elevation vs time. Once this is done, the rate of change per unit
 time is obtained from a linear regression, representing an estimate of the value of GIA between
 the two sites.
\newpage

\section{Introduction}
 The Earths crust rests on top of the mantle, its elevation rising and falling
 with the amount of mass weighing on it. During glacial periods, a significant portion
 of the water on earth is transferred in form from water in the Oceans to glacial ice,
 weighing down the continental crust. This causes the crust to ride lower in elevation,
 a change which is quickly reversed when the weight is removed as the ice sheets melt.
 This vertical motion of the crust while returning to its previous position is known
 as glacial isostatic adjustment (GIA).\\
 
 This process of isostatic rebound has implications for the routes that the flow
 of water on the Earths surface takes; the "tilting" of the surface caused by 
 uneven rates of GIA in different locations may open or close locations along basins,
 causing some rivers and lake outlets to close, while potentially opening others.
 Additionally, the change in "tilt" has potential to change shorelines of existing
 basins, which has implications for property assessment, structural projections, etc.\\
 
 In order to project the future impact of this process on the Great Lakes Basin,
 an estimate of the historical rate of GIA is needed. This estimate is obtained by
 comparing the elevation of the water mark at two different locations around a basin, and
 observing how this difference changes over time. The elevation of the water can be inferred
 by a variety of indicators in the sediment record, in this case, beach deposits known
 as strandplain sequences are used, their ages determined by optically stimulated
 luminescence (OSL) dating. This raw data is presented in Figure \ref{fig:rawData}.\\
\begin{figure}[h]
	\includegraphics[width=1.1\linewidth]{data/theDataRaw.png}
	\caption{Current day elevation of relict shorelines with respect to time before present over the last 5000 years. Strandplain sites Au Train Bay, Michigan (ATB), Batchawana Bay, Ontario (BATB), Tahquamenon Bay, Michigan (TAHB), and Grand Traverse Bay, Michigan (GTB) surrounding Lake Superior are plotted individually. Data from Johnston et al, (2012)}
	\label{fig:rawData}
\end{figure} 
%\begin{figure}[h]
	\includegraphics[width=0.40\textwidth]{data/legendary.png}
	%\makebox[\textwidth]{\includegraphics[width=\paperwidth]{data/legendary.png}}
	%\caption{Site Legend}
	%\label{fig:rdmLegend}
\end{figure}



\newpage 
 
 Given that none of the datasets have elevations sampled
 at the same times, an estimate of elevation is needed for times where one dataset
 has a data point present, but the other does not. This estimate 
 (the modelled elevation) is created by using linear interpolation between datapoints,
 represented as a solid line between points in Figure \ref{fig:rawDataWithModel}.\\
\begin{figure}[h]
	\includegraphics[width=1.1\linewidth]{data/theData.png}
	\caption{Water surface elevation with respect to time before present, modelled}
	\label{fig:rawDataWithModel}
\end{figure}
\newpage
%\begin{figure}[h]
	\includegraphics[width=0.40\textwidth]{data/legendary.png}
	%\makebox[\textwidth]{\includegraphics[width=\paperwidth]{data/legendary.png}}
	%\caption{Site Legend}
	%\label{fig:rdmLegend}
\end{figure}


 
\newpage

\section{Previous Work} 
-Current day Water gauge data used to create monthly means of water level
-Used an algorithm to adjust data for monthly biases 


\newpage
\begin{figure}[t]
	\makebox[\textwidth]{\includegraphics[width=0.72\paperwidth]{data/ATB-BATB_DataAndModel.png}}
	\caption{ATB-BATB raw data with linear interpolation model}
	\label{fig:data_ATBxBATB}
\end{figure}
\newpage

\begin{figure}[t]
	\includegraphics[width=0.9\linewidth]{data/gias/theGIA_ATB_relative_to_BATB.png}
	\caption{Differences in elevation measured from the ATB data to the ATB model}
	\label{fig:gias_ATBxBATB}
\end{figure}
\newpage


\begin{figure}[t]
	\includegraphics[width=0.9\linewidth]{data/gias/theGIA_BATB_relative_to_ATB.png}
	\caption{Differences in elevation measured from the BATB data to the ATB model}
	\label{fig:gias_BATBxATB}
\end{figure}
\newpage
% this desperately needs to be done as a loop



\begin{figure}[h]
	\makebox[\textwidth]{\includegraphics[width=0.72\paperwidth]{data/TAHB-BATB_DataAndModel.png}}
	\caption{TAHB-BATB raw data with linear interpolation model}
	\label{fig:data_TAHBxBATB}
\end{figure}
\newpage

\begin{figure}[h]
	\includegraphics[width=0.9\linewidth]{data/gias/theGIA_TAHB_relative_to_BATB.png}
	\caption{Differences in elevation measured from the TAHB data to the BATB model}
	\label{fig:gias_TAHBxBATB}
\end{figure}
\newpage


\begin{figure}[h]
	\includegraphics[width=0.9\linewidth]{data/gias/theGIA_BATB_relative_to_TAHB.png}
	\caption{Differences in elevation measured from the BATB data to the TAHB model}
	\label{fig:gias_BATBxTAHB}
\end{figure}
\newpage








\begin{figure}[h]
	\makebox[\textwidth]{\includegraphics[width=0.72\paperwidth]{data/TAHB-ATB_DataAndModel.png}}
	\caption{TAHB-ATB raw data with linear interpolation model}
	\label{fig:data_TAHBxATB}
\end{figure}
\newpage

\begin{figure}[h]
	\includegraphics[width=0.9\linewidth]{data/gias/theGIA_TAHB_relative_to_ATB.png}
	\caption{Differences in elevation measured from the TAHB data to the ATB model}
	\label{fig:gias_TAHBxATB}
\end{figure}
\newpage


\begin{figure}[h]
	\includegraphics[width=0.9\linewidth]{data/gias/theGIA_ATB_relative_to_TAHB.png}
	\caption{Differences in elevation measured from the ATB data to the TAHB model}
	\label{fig:gias_ATBxTAHB}
\end{figure}
\newpage






\begin{figure}[h]
	\makebox[\textwidth]{\includegraphics[width=0.72\paperwidth]{data/GTB-BATB_DataAndModel.png}}
	\caption{GTB-BATB raw data with linear interpolation model}
	\label{fig:data_GTBxBATB}
\end{figure}
\newpage

\begin{figure}[h]
	\includegraphics[width=0.9\linewidth]{data/gias/theGIA_GTB_relative_to_BATB.png}
	\caption{Differences in elevation measured from the GTB data to the BATB model}
	\label{fig:gias_GTBxBATB}
\end{figure}
\newpage


\begin{figure}[h]
	\includegraphics[width=0.9\linewidth]{data/gias/theGIA_BATB_relative_to_GTB.png}
	\caption{Differences in elevation measured from the BATB data to the GTB model}
	\label{fig:gias_BATBxGTB}
\end{figure}
\newpage









\begin{figure}[h]
	\makebox[\textwidth]{\includegraphics[width=0.72\paperwidth]{data/GTB-ATB_DataAndModel.png}}
	\caption{GTB-ATB raw data with linear interpolation model}
	\label{fig:data_GTBxATB}
\end{figure}
\newpage

\begin{figure}[h]
	\includegraphics[width=0.9\linewidth]{data/gias/theGIA_GTB_relative_to_ATB.png}
	\caption{Differences in elevation measured from the GTB data to the ATB model}
	\label{fig:gias_GTBxATB}
\end{figure}
\newpage


\begin{figure}[h]
	\includegraphics[width=0.9\linewidth]{data/gias/theGIA_ATB_relative_to_GTB.png}
	\caption{Differences in elevation measured from the ATB data to the GTB model}
	\label{fig:gias_ATBxGTB}
\end{figure}
\newpage










\begin{figure}[h]
	\makebox[\textwidth]{\includegraphics[width=0.72\paperwidth]{data/GTB-TAHB_DataAndModel.png}}
	\caption{GTB-TAHB raw data with linear interpolation model}
	\label{fig:data_GTBxTAHB}
\end{figure}
\newpage

\begin{figure}[h]
	\includegraphics[width=0.9\linewidth]{data/gias/theGIA_GTB_relative_to_TAHB.png}
	\caption{Differences in elevation measured from the GTB data to the TAHB model}
	\label{fig:gias_GTBxTAHB}
\end{figure}
\newpage


\begin{figure}[h]
	\includegraphics[width=0.9\linewidth]{data/gias/theGIA_TAHB_relative_to_GTB.png}
	\caption{Differences in elevation measured from the TAHB data to the GTB model}
	\label{fig:gias_TAHBxGTB}
\end{figure}
\newpage


\begin{figure}[h]
	\makebox[\textwidth]{\includegraphics[width=0.6\paperwidth]{data/intervals.png}}
	\caption{95p Confidence intervals on GIA rates obtained from site comparisons}
	\label{fig:intervalsGIA}
\end{figure}


\newpage








\end{document}
