\subsection{Previous Work} 
Mainville \& Craymer (2005) used water gauge data collected around the Great
 Lakes over the past 150 years to
 create monthly means of water level. Differences in these values between sites
 would then be plotted against time to get a rate of elevation change between
 sites over time (ie GIA). However, combinations of sites were shown to produce
 inconsistent results, so a second method using a least squares adjustment process was used,
 iteratively removing some monthly mean outliers which fell some arbitrary residual distance or
 farther from the linear regression line until none remained "too far away" from
 the final regression. A third, and ultimately best method was developed by
 Mainville \& Craymer by using the original method of directly comparing monthly
 water level means, but applying adjustments for the epoch, site, and month of each
 monthly water mean when subtracting between sites. Their findings showed better agreement with the ICE-3G
 global model of GIA than ICE-4G (Mainville \& Craymer, 2005).\\ \\
Johnston et al. (2012) attempted to refine previous estimates made using water
 gauge data by using data over a much longer timescale. In this method, water
 levels were inferred from the elevation of beach deposits from the late Holocene
 sediment record around Lake Superior, the ages for each data point measured by
 dating samples from these beach deposits (known as strandplain sequences) with
 Optically Stimulated Luminescence (OSL) age dating. This data differed from
 that used by Mainville \& Craymer in that data collected did not have
 elevations sampled at the same points in time for calculation of relative
 rates. As a result, the elevation vs time data was modelled with a linear
 regression for each site, the difference in slopes representing the GIA rate
 between sites (Johnston et al, 2012). In a later 2014 paper, Johnston et al.
 attempted to refine the method by adjusting the model of each site upward or 
 downwards with common lake level lows and highs observed in the other sites
 (Johnston et al, 2014).
 % need to ask JJ about JJ 2012, bottom of page 3, divergence of intercepts
 
