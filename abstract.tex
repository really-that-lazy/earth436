


\DTLloaddb{result}{data/bothNonZero_withinSeventyFivePercent_intervals.csv}
\newcommand{\result}[1]{%
    \dtlgetrowforvalue{result}{\dtlcolumnindex{result}{id}}{#1}%
    % read variables into commands
\dtlgetentryfromcurrentrow{\id}{1}
\dtlgetentryfromcurrentrow{\name}{2}
\dtlgetentryfromcurrentrow{\startValue}{3}
\dtlgetentryfromcurrentrow{\endValue}{4}

% use commands in text
\text{\startValue} to \text{\endValue} cm/century

%
}


\section{Abstract}

The ground surface underlying the Laurentian Great Lakes is currently undergoing vertical adjustment
 after being depressed by the weight of an ice sheet formed in the most recent glacial period during the Wisconsonian.
 The rate of glacial isostatic adjustment (GIA) varies by location, and strongly influences the flow of water
 in the Laurentian Great Lakes (LGL) as the inclination of the ground surface changes. In order to predict the
 future movement in the LGL, the rate of GIA must be
 inferred from measurements of the water level in the geological record. These measurements are
 made by measuring the elevation of a subsurface sedimentary contact relating to
 past lake levels,which are then age dated with optically stimulated luminescence
 (OSL) to provide an age for sediments. Elevation and age data are then compiled
 to create site paleohydrographs for each location around the lake basin.\\
 
 The focus of this paper is to analyze the data compiled by Johnston et al, 2014
 which measured past elevation of shorelines with the OSL method. Each site paleohydrograph
 is now extended between points where the data was measured directly, then subtracted
 from one data point to a modelled elevation in order to create a
 plot of relative elevation over time. Once this is done, the rate of change per unit
 time is obtained from a linear regression, representing an estimate of the value
 of GIA between each pair of sites. This process is repeated for
 all possible combinations of the four sites used, 
 Grand Traverse Bay (GTB), Au Train Bay (ATB), Batchawana Bay (BATB), and Tahquamenon Bay (TAHB).\\
 
 The results of this process were in strong agreement at the 95 \% confidence level for GIA rates obtained from
 forward and reverse regressions for the combination of ATB-BATB (23.5 to 31 cm/century) and
 BATB-TAHB (11 to 17 cm/century). Agreement was also seen at the 95 \% confidence level for GTB-TAHB (anywhere from -3 to 8.5 cm/century),
 ATB-GTB (9 to 13 cm/century) and ATB-TAHB (19.5 to 29 cm/century).
