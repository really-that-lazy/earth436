


\DTLloaddb{result}{data/bothNonZero_withinSeventyFivePercent_intervals.csv}
\newcommand{\result}[1]{%
    \dtlgetrowforvalue{result}{\dtlcolumnindex{result}{id}}{#1}%
    % read variables into commands
\dtlgetentryfromcurrentrow{\id}{1}
\dtlgetentryfromcurrentrow{\name}{2}
\dtlgetentryfromcurrentrow{\startValue}{3}
\dtlgetentryfromcurrentrow{\endValue}{4}

% use commands in text
\text{\startValue} to \text{\endValue} cm/century

%
}


\section{Abstract}

The ground surface underlying the Laurentian Great Lakes is currently undergoing vertical adjustment
 after being depressed by the weight of an ice sheet formed in the most recent glacial period during the Wisconsonian.
 The rate of glacial isostatic adjustment (GIA) varies by location, significantly influencing the flow of water
 in the Laurentian Great Lakes (LGL) as the inclination of the ground surface changes at different rates
 in different locations. Previous attempts to 
 estimate the rate of GIA between sites used water gauge data from the
 past 150 years in order to measure the rate of this long term geologic process. In contrast, by
 inferring GIA from measurements of past water levels as preserved in the geological record over the past 5000 years,
 a more accurate estimate of the long term process of GIA can be obtained. These
 datasets were sampled by measuring the elevation of a subsurface sedimentary contact relating to
 past lake levels, which are then age dated using optically stimulated luminescence
 (OSL) to provide an age for sediments. Elevation and age data are then compiled
 to create site paleohydrographs for each location around the lake basin that are
 compared to resolve a measurement for relative GIA between study sites.\\
 
 The focus of this paper is to analyze the elevation and age data compiled by Johnston et al, 2012
 which measured past elevation of shorelines by interpreting water levels recorded
 in the sediment record. In order to deal with the unequal spacing of data from
 different sites, the datapoints in each site paleohydrograph
 are first linearly interpolated between points where the data was measured directly, then subtracted
 from data points to a modelled elevations in order to create a
 plot of difference in relative elevation over time. Once this is done, the rate of change per unit
 time is obtained from a linear regression, representing the rate
 of GIA between each pair of sites. This process is repeated for
 all possible combinations of the four study sites of Johnston et al. (2012) around Lake Superior, 
 Grand Traverse Bay (GTB), Au Train Bay (ATB), Batchawana Bay (BATB), and Tahquamenon Bay (TAHB).\\
 
 
 
 The results of this process were agreement at the 95\% confidence level for GIA rates obtained from
 forward and reverse regressions for the combination of ATB-BATB (23.5 to 31 cm/century) and
 BATB-TAHB (11 to 17 cm/century). Agreement was also seen at the 95 \% confidence level for GTB-TAHB (anywhere from -3 to 8.5 cm/century),
 ATB-GTB (9 to 13 cm/century), GTB-BATB (10.5 to 12 cm/century), and ATB-TAHB (19.5 to 29 cm/century).
