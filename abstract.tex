\section{Abstract}
The ground surface underlying the Laurentian Great Lakes is currently undergoing vertical adjustment
 after being depressed by the weight of an ice sheet formed in the most recent glacial period.
 The rate of glacial isostatic adjustment (GIA) varies by location, and exerts a significant control on the flow of water
 in the Laurentian Great Lakes as the inclination of the ground surface changes. In order to predict the
 future movement in this area, the rate of GIA must be
 inferred from measurements of the water level in the geological record. These measurements are
 made by measuring the elevation of a subsurface sedimentary contact relating to
 past lake levels,which are then age dated with optically stimulated luminescence
 (OSL) to provide an age value.\\
 
 The focus of this paper is to analyze this data by measuring the relative difference in water levels
 between study sites, comparing differences in relative water levels to create a
 plot of relative elevation over time. Once this is done, the rate of change per unit
 time is obtained from a linear regression, representing an estimate of the value
 of GIA between each pair of sites. This is done for
 all possible combinations of the four sites used, 
 Grand Traverse Bay (GTB), Au Train Bay (ATB), Batchawana Bay (BATB), and Tahquamenon Bay (TAHB).\\
 
 The results of this process were a strong agreement of 95p confidence intervals on GIA rates obtained from
 forward and reverse regressions for the combination of BATB-GTB (difference of 0.012 m/year) and
 BATB-ATB (0.028 m/year). Agreement was also seen for GTB-TAHB (range of 0.0005-0.0007 m/year),
 ATB-GTB (0.0011-0.0014 m/year), and ATB-TAHB (0.0020-0.0027 m/year), while the rate for BATB-TAHB
 strongly disagreed between forward and reverse comparisons, making estimation of
 GIA from that combination of sites unreliable.
 
